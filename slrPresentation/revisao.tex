%%%%%%%%%%%%%%%%%%%%%%%%%%%%%%%%%%%%%%%%%%%%%%%%%%%%%%%%%%%%%%%%%%%%%%%%%%%%%%%%%
% Beamer presentationtion 															
% LaTeX Template 																
% Version 1.0 (10/11/12)														%
% This template has been downloaded from: http://www.LaTeXTemplates.com 		%
% https://www.sharelatex.com/blog/2013/08/14/beamer-series-pt2.html
% https://www.writelatex.com/docs?snip_uri=http://www.latextemplates.com/templates/presentations/1/presentation_1.zip				
% License: CC BY-NC-SA 3.0 (http://creativecommons.org/licenses/by-nc-sa/3.0/)  
% Compilar: pdflatex ficheiroTex.tex
%%%%%%%%%%%%%%%%%%%%%%%%%%%%%%%%%%%%%%%%%%%%%%%%%%%%%%%%%%%%%%%%%%%%%%%%%%%%%%%%%

%----------------------------------------------------------------------------------------
%	PACKAGES AND THEMES
%----------------------------------------------------------------------------------------

\documentclass[10pt]{beamer}

\mode<presentation> {

% The Beamer class comes with a number of default slide themes
% which change the colors and layouts of slides. Below this is a list
% of all the themes, uncomment each in turn to see what they look like.

%\usetheme{default}
%\usetheme{AnnArbor}
%\usetheme{Antibes}
%\usetheme{Bergen}
%\usetheme{Berkeley}
%\usetheme{Berlin}
%\usetheme{Boadilla}
%\usetheme{CambridgeUS}
%\usetheme{Copenhagen}
%\usetheme{Darmstadt}
%\usetheme{Dresden}
%\usetheme{Frankfurt}
%\usetheme{Goettingen}
%\usetheme{Hannover} OK
%\usetheme{Ilmenau} OK
%\usetheme{JuanLesPins}
%\usetheme{Luebeck} OK
%\usetheme{Madrid}
%\usetheme{Malmoe}
%\usetheme{Marburg}
%\usetheme{Montpellier}
%\usetheme{PaloAlto}
%\usetheme{Pittsburgh}
%\usetheme{Rochester}
%\usetheme{Singapore}
%\usetheme{Szeged}
%\usetheme{Warsaw}
\usetheme[progressbar=frametitle]{metropolis}
% As well as themes, the Beamer class has a number of color themes
% for any slide theme. Uncomment each of these in turn to see how it
% changes the colors of your current slide theme.

%\usecolortheme{albatross}
%\usecolortheme{beaver}
%\usecolortheme{beetle}
%\usecolortheme{crane}
%\usecolortheme{dolphin}
%\usecolortheme{dove}
%\usecolortheme{fly}
%\usecolortheme{lily}
%\usecolortheme{orchid}
%\usecolortheme{rose}
%\usecolortheme{seagull}
%\usecolortheme{seahorse}
%\usecolortheme{whale}
%\usecolortheme{wolverine}

%\setbeamertemplate{footline} % To remove the footer line in all slides uncomment this line
%\setbeamertemplate{footline}[page number] % To replace the footer line in all slides with a simple slide count uncomment this line

%\setbeamertemplate{navigation symbols}{} % To remove the navigation symbols from the bottom of all slides uncomment this line
}
\usepackage{appendixnumberbeamer}
\usepackage{booktabs}
\usepackage[scale=2]{ccicons}
\usepackage{pgfplots}
\usepgfplotslibrary{dateplot}
\usepackage{xspace}
\newcommand{\themename}{\textbf{\textsc{metropolis}}\xspace}
%\usepackage{graphicx} % Allows including images
%\usepackage{booktabs} % Allows the use of \toprule, \midrule and \bottomrule in tables
%\usepackage[utf8]{inputenc} %Permite a utilização dos carateres portugueses
\usepackage{verbatim} %Permite inserir os comentários
%\usepackage{graphicx} %Permite inserir gráficos
%\usepackage[portuguese]{babel} % Para termos texto em português, e.g. legendas e datas
%\usepackage[english]{babel}
%\setbeamertemplate{caption}[numbered] %para numerar os captions
%\usepackage[labelfont=scriptsize,labelfont=bf,font=scriptsize,belowskip=-15pt,aboveskip=5pt]{caption}

%----------------------------------------------------------------------------------------
%	Otimização da gestão de instalações desportivas nas empresas municipais
%   Aplicação de uma abordagem baseada na gestão por processos
%----------------------------------------------------------------------------------------

\title[Dropout Prediction]{Dropout Prediction: A Systematic Literature Review} 
\subtitle{SLR Dropout} 
% The short title appears at the bottom of every slide, the full title is only on the title page

\begin{comment} % Estrutura da apresentação

\end{comment}

\author{Pedro Sobreiro, José Garcia Alonso, Javier Berrocal} % Your name
\institute[UNEX] % Your institution as it will appear on the bottom of every slide, may be shorthand to save space
{ 
University of Extremadura ~~~~~~~~~~~~~~~% Your institution for the title page
\medskip
\textit{pesobreiro@gmail.com} % Your email address
}
\date{Webinar, 26 June 2020} % Date, can be changed to a custom date

\begin{document}

\begin{frame}
	\titlepage % Print the title page as the first slide
\end{frame}

\begin{frame}
\frametitle{Summary} % Table of contents slide, comment this block out to remove it
\tableofcontents % Throughout your presentation, if you choose to use \section{} and \subsection{} commands, these will automatically be printed on this slide as an overview of your presentation

\end{frame}

%----------------------------------------------------------------------------------------
%	PRESENTATION SLIDES
%----------------------------------------------------------------------------------------

%------------------------------------------------
\section{Research goals} % Sections can be created in order to organize your presentation into discrete blocks, all sections and subsections are automatically printed in the table of contents as an overview of the talk
%------------------------------------------------
\begin{comment}
Dropout predicting is challenging analysis process which requires appropriate approaches to address the dropout. Existing approaches are applied in different areas such as education, telecommunications, retail, social networks, and banking services. The goal is to identify customers in the risk of dropout to support retention strategies. This research developed a systematic literature review to evaluate the development of existing studies to predict dropout using machine learning,  following the guidelines recommended by Kitchenham and Peterson. The systematic review followed three phases planning, conducting and reporting. The selection of the most relevant articles was based on the use of Active Systematic Review tool using artificial intelligence algorithms. The criteria identified 28 articles and several research lines where identified. Dropout is a transversal problem for several sectors of economic activity, where it can be taken countermeasures before it happens if detected early.
\end{comment}

\begin{frame}
	\frametitle{Objetivo do estudo}
	% Vamos colocar os objetivos do estudo 
	\Large{Compreender a perceção dos professores do ensino superior em atividades de ensino online numa situação de emergência provocada pelo COVID-19 e obter indicadores para auxiliar na tomada de decisão.}

	
\end{frame}

\section{Introduction} % A subsection can be created just before a set of slides with a common theme to further break down your presentation into chunks
\begin{comment}
Esta investigação, desenvolvida através da técnica de inquérito por questionário, pretende avaliar como é que esses docentes encararam essa realidade. O trabalho de campo foi realizado na primeira quinzena de abril de 2020 envolvendo docentes de uma instituição de ensino superior portuguesa tendo-se obtido 35 respostas válidas.

\end{comment}

\begin{frame}
	\frametitle{Introduction}
	\Large
	\textbf{Ensino online}\\
		\begin{itemize} \normalsize
			\item Customer analysis is fundamental to develop business and marketing intelligence \footnotesize(Sheth, Mittal, \& Newman, 1998)\normalsize

		\end{itemize}	

	
	
	\tiny
	~~~~Allen, I. E., \& Seaman, J. (2011). Going the Distance: Online Education in the United States, 2011. Em Sloan Consortium (NJ1). Obtido de https://eric.ed.gov/?id=ED529948 \\
	~~~~Martinho, D. S. (2014). O Ensino Online nas Instituições de Ensino Superior Privado. As perspetivas: Docente e discente e as implicações na tomada de decisão institucional.\\
	~~~~McCarthy, S. A. (2009). Online Learning as a Strategic Asset. Volume I: A Resource for Campus Leaders. A Report on the Online Education Benchmarking Study Conducted by the APLU-Sloan National Commission on Online Learning. Em Association of Public and Land-grant Universities. Obtido de https://eric.ed.gov/?id=ED517308\\

\end{frame}

\section{Metodologia}
\begin{comment}
The data was anonymised, removing all personal information before being retrieved from the management system.
Esta investigação, desenvolvida através da técnica de inquérito por questionário, pretende avaliar como é que esses docentes encararam essa realidade. O trabalho de campo foi realizado na primeira quinzena de abril de 2020 envolvendo docentes de uma instituição de ensino superior portuguesa tendo-se obtido 35 respostas válidas.
\end{comment}
\begin{frame}
	\frametitle{Metodologia}
	\begin{itemize}
		\item Foram tratados dados de 34 docentes de uma instituição de ensino superior;
		\item O estudo está numa fase exploratória e está a ser desenvolvido. Utilizamos um questionário validado (Martinho, 2014);
		\item O tratamento de dados foi realizado com o Anaconda e IPython \footnotesize(Continuum Analytics, 2016)\normalsize ~e recorrendo ao Pandas \footnotesize(McKinney \& others, 2010) \normalsize e NumPy \footnotesize(Walt, Colbert, \& Varoquaux, 2011)\normalsize;
		\item Realizamos o teste de normalidade às variáveis, onde não detetamos uma distribuição normal com p$<$0.05. As correlações foram realizadas com o coeficiente Spearman;



	\end{itemize}
	\tiny 
	~~~~Martinho, D. S. (2014). O Ensino Online nas Instituições de Ensino Superior Privado. As perspetivas: Docente e discente e as implicações na tomada de decisão institucional.\\
	~~~~~Continuum Analytics. (2016). Anaconda Software Distribution. Obtido 20 de Julho de 2017, de https://www.anaconda.com/download/ \\
	~~~~~McKinney, W., \& others. (2010). Data structures for statistical computing in python. Em Proceedings of the 9th Python in Science Conference (Vol. 445, pp. 51–56). SciPy Austin, TX. Obtido de https://pdfs.semanticscholar.org/f6da/c1c52d3b07c993fe52513b8964f86e8fe381.pdf\\
	~~~~~Walt, S. van der, Colbert, S. C., \& Varoquaux, G. (2011). The NumPy Array: A Structure for Efficient Numerical Computation. Computing in Science \& Engineering, 13(2), 22–30. doi:10.1109/MCSE.2011.37


\end{frame}


\section{Resultados}
\begin{comment} 
\begin{frame}
	\frametitle{Determinação do número de clusters}
	\begin{figure}
		\includegraphics[scale=0.25]{images/elbowAnalysis.png}
		\caption{Representação para identificação visual do número de clusters}
		\label{figure1}
	\end{figure}
\end{frame}

\begin{frame}
	\frametitle{Representação dos dados 2D}
	\begin{figure}
		\includegraphics[scale=0.25]{images/clusters2D.png}
		\caption{Reduzimos a as variáveis a duas dimensões recorrendo ao \emph{Principal Components Analysis} para facilitar a representação}
		\label{figure2}
	\end{figure}
\end{frame}
\end{comment}

\begin{frame}[fragile]{Resultados}
  	\begin{itemize}
		\item Os resultados revelam que a maioria dos professores que responderam ao questionário são do sexo masculino (56\%)
		\item Questionados sobre a avaliação da qualidade do ensino online versus ensino presencial 24\% considera que é inferior, 44\% refere que não existem diferenças e 26\% por vezes é superior
		\item Quando confrontados com gosto pelo ensino online em relação ao seu gosto pelo ensino presencial 6\% considera inferior, 29\% por vezes inferior, 21\% não apresenta preferência, 38\% acha que é por vezes superior e 6\% considera superior
		\item Em relação à disponibilidade para o ensino online em relação à sua disponibilidade para o ensino presencial 18\% apresenta mais disponibilidade para o ensino online, 41\% considera a sua disponibilidade por vezes superior, 35\% é indiferente e 6\% considera ter menos disponibilidade.
	\end{itemize}
\end{frame}

\begin{frame}[fragile]{Resultados}
  	\begin{itemize}
		\item Quando questionados sobre os aspetos tecnológicos, 100\% apresenta experiência de mais de 6 anos na utilização processadores de texto e email. 97\% PowerPoint e motores de busca;
		\item Em relação a ferramentas de videoconferência verifica-se que 56\% dos respondentes têm mais de 6 anos de experiência com o Zoom, 21\% de 3 a 6 anos e 24\% até 3 anos de experiência
		\item Na utilização de ferramentas para chat 76\% mais de 6 anos, 15\% de 3 a 6 anos, 6\% menos de 3 e 3\% não apresentava nenhuma experiência
		\item Na utilização do Moodle 53\% apresentava mais de 6 anos de experiência, 29\% 3 a 6 anos e 18\% menos de 3 anos
	\end{itemize}
\end{frame}

\begin{frame}[fragile]{Resultados}
  	\begin{itemize}
		\item A análise de correlação mostrou relações com elevado significado estatístico entre a qualidade do ensino online versus ensino presencial* e: (1) gosto em ensinar presencialmente (r\textsubscript{s}=.51,p$<$.01); (2) disponibilidade para o ensino online (r\textsubscript{s}=.44,p$<$.01); (3) preferência por testes online (r\textsubscript{s}=.46,p$<$.01); (4) preferência com apresentações orais(r\textsubscript{s}=.40,p$<$.05); (5) experiência com chat (r\textsubscript{s}=.36,p$<$.05);
		\item Não se identificaram relações significativas entre qualidade do ensino online e: (1) idade, (2) anos docência, (2) satisfação com Moodle, Zoom e Google Docs;
	\end{itemize}
	\tiny * Escala de likert com as opções: 1-Inferior; 2- Por vezes inferior; 3 – Sem diferenças significativas ; 4-Por vezes superior;  5-Superior 

\end{frame}


\section{Discussão e conclusões}
\begin{frame}
	\frametitle{Discussão e Conclusões}
	\begin{itemize}

 		\item A dimensão da amostra não permite a generalização dos resultados obtidos pelo que o trabalho de campo vai continuar com o alargamento da amostra dos respondentes para outras instituições do ensino superior;

		\item Pretende-se ainda investigar outros aspetos, nomeadamente como é que os professores perspetivam o ensino online para o desenvolvimento dos estudantes, da sua carreira profissional e das instituições de ensino superior;


	\end{itemize}
	\tiny 
\end{frame}


\begin{frame}
\frametitle{Obrigado!}
\normalsize
	Pedro Sobreiro - IPSantarém  \\~\\
	Domingos Martinho - ISLA Santarém \\~\\
	Marco Tereso - ISLA Santarém \\~\\
\end{frame}


\end{document}\grid
